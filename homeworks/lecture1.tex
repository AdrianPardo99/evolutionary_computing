\documentclass[12pt]{article}
\usepackage[utf8]{inputenc}
\usepackage[spanish]{babel}
\usepackage{amsmath}
\usepackage{amsfonts}
\usepackage{amssymb}
\usepackage{graphics}
\usepackage{graphicx}
\usepackage[left=2cm,right=2cm,top=2cm,bottom=2cm]{geometry}
\usepackage{imakeidx}
\makeindex[columns=3, title=Alphabetical Index, intoc]
\usepackage{listings}
\usepackage{xcolor}
\usepackage{multicol}
\usepackage{changepage}
\usepackage{float}
\usepackage{cite}
\usepackage{url}
\usepackage{hyperref}
\usepackage[document]{ragged2e}
\hypersetup{
    colorlinks=true,
    linkcolor=blue,
    filecolor=magenta,
    urlcolor=blue,
}

\definecolor{codegreen}{rgb}{0,0.6,0}
\definecolor{codegray}{rgb}{0.5,0.5,0.5}
\definecolor{codepurple}{rgb}{0.58,0,0.82}
\definecolor{backcolour}{rgb}{0.95,0.95,0.92}

\lstdefinestyle{mystyle}{
    backgroundcolor=\color{backcolour},
    commentstyle=\color{codegreen},
    keywordstyle=\color{magenta},
    numberstyle=\tiny\color{codegray},
    stringstyle=\color{codepurple},
    basicstyle=\ttfamily\footnotesize,
    breakatwhitespace=false,
    breaklines=true,
    captionpos=b,
    keepspaces=true,
    numbers=left,
    numbersep=5pt,
    showspaces=false,
    showstringspaces=false,
    showtabs=false,
    tabsize=3
}

\lstset{style=mystyle}

\title{}

\author{Instituto Politécnico Nacional\\Escuela Superior de Computo\\Bioinformatics\\Gonzalez Pardo Adrian\\Rosas Trigueros Jorge Luis\\El concepto de Evolución\\Relase date: \today}

\date{Delivery date: August 23rd, 2021}

\newcommand\tab[1][1cm]{\hspace*{#1}}

\begin{document}
\maketitle
\section{Preguntas}
\begin{enumerate}
  \item ¿Qué es la evolución?
  \item ¿Qué es un sistema evolutivo?
  \item Proporcione un ejemplo de la evolución de un sistema (Ilustre)
  \item ¿La evolución consigue que un sistema mejore?
  \item Señale dos errores de ortografía/gramática en el texto
\end{enumerate}
\section{Respuestas}
\begin{enumerate}
  \item Se puede definir como la constante adaptación natural de un ser o sistema el cual busca seguir coexistiendo con su medio e interactuar con los demás seres o sistemas.
  \item Es un sistema que guarda relación o una equivalencia a un modelo/sistema natural, en el cual a través de su interacción con el medio que lo rodea busca adaptarse para su supervivencia, en algunos casos esta adaptación implica grandes cambios que en términos de medición puede tratarse de situaciones en caos, de igual forma este sistema tiene varias etapas que trabajan de forma conjunta y a la vez de manera paralela, tal es el caso que explica el crecimiento y el desarrollo de un ser, donde bien puede acabarse un proceso pero los otros siguen existiendo o pueden comenzar a trabajar.
  \item Un proceso claro de la evolución de un sistema en múltiples ramas pero que comúnmente solo engloban en una sola rama es el caso de la evolución del hombre, ya que este no solo partió de un caso muy primitivo hace miles de años, sino que este comenzó a desarrollar su sistema tanto de manera física, como de manera psicológica, ya que involucro su misma curiosidad y su ansia por sobrevivir al medio que estaba en constante cambio geográfico y ambiental, de modo en que paso de ser una criatura cuya complexión era andar en 4 patas a ser un ser erguido que anda a dos patas y tiene la habilidad de crear innovar, buscar y explotar su ingenio yendo desde la era de la prehistoria, de las primeras colonias, donde no solo se vieron involucrados con otras facciones que venían de la misma especie, sino que buscaron adaptarse a nuevos cambios, creando y estudiando los sistemas en su sed de conocimiento y por entender el mundo, creando guerras, saciando sus más bajos instintos, viendo como cada individuo es tan único y complejo que guardan patrones de similitudes.
  \item El sistema mejora en cuanto a las condiciones de sobrevivir al medio, pero esto no siempre significa que este cambio sea el mejor.
  \item Párrafo:\\
    "Tal vez esta es una de las causas por las que el cerebro requiere un espacio 'no usado' (se dice que solo usamos el diez por ciento de nuestra capacidad) ya que tal vez no es tan vacío o no usado como se cree, sino que actúa como un gran césped donde todo lo que pasa se graba aunque sea muy tenuemente y en su momento algunos aspectos se marcan y otros desaparecen y permanentemente esta cambiando la forma de ese espacio"
    \\Puede reescribirse o al menos usar una coma entre cada \textbf{y} ya que esto puede representar un abuso de la misma
    \\
    En algunos otros casos se hace uso de la palabra \textbf{mas} a la cual en solo algunos párrafos si es necesario que use la palabra con tilde ya que se explica como una idea comparativa.
\end{enumerate}

\end{document}
